\newword{30}{Função fitness}{Função que pode medir o quão boa é uma solução em uma escala numérica}

\newword{35}{Java}{Linguagem de programação orientada a objetos que gera um \textit{bytecode} para ser interpretado pela JVM, portanto, é uma linguagem semi-compilada, trazendo benefícios de linguagens compiladas e interpretadas}

\newword{36}{Java Virtual Machine}{JVM é uma máquina virtual que possibilita um computador executar programas em Java}

\newword{37}{AE}{Algoritmo Evolutivo. Algoritmos relacionados à Computação Evolutiva. Englobam os algoritmos Genéticos, mas não se resumem a estes. Inspirados na Teoria da Evolução Natural de Charles Darwin}

\newword{38}{Cromossomo}{Solução candidata, utilizada pelo Algoritmo Evolutivo em seu processo de busca}

\newword{39}{Crossover}{Mecanismo da biologia abstraído na forma de uma função matemática utilizada pelo Algoritmo Evolutivo, aplicada sobre os cromossomos. Faz com que uma nova solução possua características de soluções anteriores}

\newword{40}{Adaptação}{Melhoria do valor de aptidão (fitness) presente no passar das gerações}

\newword{41}{Aptidão}{Pontuação que indica a eficácia da solução encontrada}

\newword{42}{Geração}{O conjunto das soluções contidas em uma população em um determinado instante; sempre que os operadores evolutivos criam uma nova população, com novas soluções, diz-se que uma nova geração foi iniciada}

\newword{43}{Mutação}{Mecanismo da biologia abstraído na forma de uma função matemática utilizada pelo Algoritmo Evolutivo, aplicada sobre os cromossomos. Insere variedade nas soluções da população}
