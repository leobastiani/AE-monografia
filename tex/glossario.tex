\newword{1}{Agentes de usuário}{qualquer \textit{software} que recupera, processa e facilita a interação do usuário final com o conteúdo  Web. Como exemplo desses agentes, podem ser citados navegadores, reprodutores multimídia e tecnologias assistivas} 

\newword{2}{ATAG}{\textit{Authoring Tool Accessibility Guidelines} ou Diretrizes de acessibilidade para ferramentas de autoria. É um conjunto de diretrizes para desenvolvedores de qualquer ferramenta de criação para Web, como: simples editores HTML, ferramentas para exportar conteúdo para Web, ferramentas multimídia e sistemas de gerenciamento de conteúdo}

\newword{3}{DOM}{\textit{Document Object Model} ou Modelo de Objetos de Documentos. É uma especificação do W3C para organizar objetos de um documento em que se pode, dinamicamente, alterar e editar sua estrutura, conteúdo e estilo}

\newword{4}{Javascript}{é uma linguagem de \textit{script} para desenvolvimento de certos tratamentos 
que ocorrem lado do cliente, geralmente o navegador Web. Ela é utilizada geralmente quando 
é inconveniente ou impossível para o servidor para fazer esse tratamento}

\newword{5}{Stakeholder}{qualquer pessoa ou grupo, que legitima as ações de uma organização. É formado pelos funcionários da empresa, gestores, gerentes, proprietários, fornecedores, clientes, o Estado, credores, sindicatos e diversas outras pessoas ou empresas que estejam relacionadas com uma determinada ação ou projeto.}

\newword{6}{Tag}{ou etiqueta, é uma palavra-chave ou termo associado com uma informação que a descreve. Em linguagens de marcação, como o HTML, consistem em breves instruções, tendo uma marca de início e outra de fim para que o navegador possa mostrar a renderização da página}

\newword{7}{Tecnologia assistiva}{Conjunto de técnicas, aparelhos, instrumentos, produtos e procedimentos que visam auxiliar a mobilidade, percepção e utilização do meio ambiente e dos elementos por pessoas com deficiência	}

\newword{8}{UAAG}{\textit{User Agent Accessibility Guidelines} ou Diretrizes de acessibilidade para agentes de usuário. Conjuntos de diretrizes para desenvolvedores de agentes de usuário (por exemplo: navegadores e reprodutores de mídia) com a finalidade de fazer com que tais agentes permitam sua utilização adequada por pessoas com algum tipo de deficiência}

\newword{9}{W3C}{\textit{World Wide Web Consortium}. É um consórcio internacional formado por empresas, órgãos governamentais e organizações independentes que visa desenvolver padrões para a criação e a interpretação de conteúdos da Web}

\newword{10}{WAI}{\textit{Web Accessibility Initiative} ou Iniciativa de Acessibilidade na Web. É a iniciativa do W3C no que tange a desenvolver estratégias, diretrizes e outros recursos, a fim de que as informações presentes na Web sejam acessíveis para pessoas com ou sem deficiência}

\newword{11}{WCAG}{\textit{Web Content Accessibility Guidelines} ou Diretrizes de Acessibilidade para Conteúdo Web. É um conjunto de diretrizes criado pelo W3C para auxílio na elaboração de conteúdo acessível, que atualmente está em sua versão 2.0 desde 2008.}

\newword{12}{WYSIWYG}{``What You See Is What You Get''  ou ``O que você vê é o que você obtém''.  Recurso tem por objetivo permitir que um documento, enquanto manipulado na tela, tenha a mesma aparência de sua utilização, usualmente sendo considerada final. Isso facilita para o desenvolvedor que pode trabalhar visualizando a aparência do documento sem precisar salvar em vários momentos e abrir em um \textit{software} separado de visualização}

\newword{13}{Desenho universal}{concepção de produtos, ambientes, programas e serviços a serem usados, na maior medida possível, por todas as pessoas, sem necessidade de adaptação ou projeto específico. O desenho universal não excluirá as ajudas técnicas para grupos específicos de pessoas com deficiência, quando necessárias}

\newword{14}{Mobilidade reduzida}{dificuldade permanente ou temporária que uma pessoa tem para se movimentar, gerando redução efetiva da mobilidade, flexibilidade, coordenação motora e percepção}

\newword{15}{Pessoas com deficiência}{aquelas que têm impedimentos de longo prazo de natureza física, mental, intelectual ou sensorial, os quais, em interação com diversas barreiras, podem obstruir sua participação plena e efetiva na sociedade em igualdades de condições com as demais pessoas. Atualmente chegou-se a um consenso quanto à utilização da expressão ``pessoa com deficiência'' em todas as suas manifestações orais ou escritas, em lugar de termos como ``deficiente'', ``pessoa portadora de deficiência'', ``pessoa com necessidades especiais'' e ``portador de necessidades especiais''}

\newword{16}{Framework}{é uma abstração que une códigos comuns entre vários projetos de \textit{software} provendo uma funcionalidade genérica. \textit{Frameworks} são projetados com a intenção de facilitar o desenvolvimento de \textit{software}, habilitando designers e programadores a gastarem mais tempo determinando as exigências do \textit{software} do que com detalhes de baixo nível do sistema}

\newword{17}{Dojo Toolkit}{é uma biblioteca em JavaScript, de código fonte aberto, projetado para facilitar o rápido desenvolvimento de interfaces ricas}

\newword{18}{Meta-tag}{Uma etiqueta HTML identificando o conteúdo de um \textit{website}. Informações comumente encontradas na meta-tag incluem: direitos autorais, palavras-chave para ferramentas de busca e descrições da formatação da página}

\newword{19}{DOCTYPE}{\textit{Document Type Definition} ou Definição do tipo de documento. Indica para o navegador e para outros meios qual a especificação de código utilizar. O DOCTYPE não é uma \textit{tag} do HTML, mas uma instrução para que o navegador tenha informações sobre qual versão de código a marcação foi escrita}

\newword{20}{Template}{é um documento sem conteúdo, com apenas a apresentação visual (apenas cabeçalhos por exemplo) e instruções sobre onde e qual tipo de conteúdo deve entrar a cada parcela da apresentação}

\newword{21}{Padrões de projeto}{ou \textit{Design Pattern}, descreve uma solução geral reutilizável para um problema recorrente no desenvolvimento de sistemas de \textit{software} orientados a objetos. Não é um código final, é uma descrição ou modelo de como resolver o problema do qual trata, que pode ser usada em muitas situações diferentes}

\newword{22}{e-MAG}{Modelo de Acessibilidade de Governo Eletrônico. Consiste em um conjunto de recomendações a ser considerado para que o processo de acessibilidade dos sítios e portais do governo brasileiro seja conduzido de forma padronizada e de fácil implementação}

\newword{23}{Web}{Sinônimo mais conhecido de \textit{World Wide Web} (WWW). É a interface gráfica da Internet que torna os serviços disponíveis totalmente transparentes para o usuário e ainda possibilita a manipulação multimídia da informação}
