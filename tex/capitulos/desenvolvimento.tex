\section{O Problema}

O desenvolvimento de um algoritmo evolutivo genérico para solucionar problemas com gamificação de disputas, ou seja, soluções que podem ser qualificadas entre o quão melhores uma é das demais.

Estudar as aplicações desse algoritmo em um sistema com entradas e saídas bem definidas, de modo que a solução não esteja implícita em nenhuma das entradas e saídas, mas que o algoritmo possa evoluir livremente para encontrar o melhor rearranjo dos genes até produzir respostas idênticas as oferecidas pelo sistema.

\section{Atividades Realizadas}

Foi feita uma biblioteca para uso de algoritmos evolutivos em Java que empregas os conceitos de Indivíduos e Genes. Para aplicações diferentes das estudadas nesse documento, a biblioteca ainda tem um uso muito bom, pois pode ser estendida e rearranjada para qualquer aplicação dentro desse conceito.

\section{Resultados}

A biblioteca produzida pode ser aplicada para quaisquer situações em que a gamificação baseada em disputas de dois jogadores pode ser aplicada.

Em sistemas cujas respostas não são determinísticas, mas sabe-se as entradas e as respostas, este algoritmo pode ser aplicado utilizando o conceito de Genes desenvolvido. Os resultados podem não ser exatos, porém haverá respostas mais consistentes do que uma resposta aleatória.

\section{Dificuldades e Limitações}

A maior dificuldade desse algoritmo, assim como da maioria dos algoritmos evolutivos, é encontrar uma função fitness capaz de representar o quão uma solução é melhor do que outra.