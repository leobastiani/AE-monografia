O algoritmo desenvolvido pode ser aplicado para diversas áreas da computação bio-inspirada e acrescenta um novo conceito sobre funções \fitness. A função \fitness pode ser obtida em casos de disputa entre dois jogadores comparando o desempenho entre eles numa disputa. Isso cria outro paradigma na produção de um \SE, entre a comparação de melhores ou piores soluções.

Outra alternativa de uso da biblioteca é encontrar uma solução que não seja a ideal de um sistema, mas sim uma solução que vença outra solução fornecida, ou seja, um possível uso da biblioteca é não confrontar indivíduos entre si, mas sim confrontar um indivíduo com outro fornecido pelo usuário como uma solução base.

Como visto nas figuras \ref{figura:resultado_97} e \ref{figura:resultado_79}, a solução encontrado pela biblioteca é fruto de um processo de convergência do algoritmo evolutivo, mesmo podendo haver resultados em que uma solução totalmente aleatória fica com valores próximos a solução do \SE.

\section{Trabalhos futuros}

Para remover a dependência da linguagem Java, a biblioteca produzida pode ser escrita em C/C++ e compilada como uma biblioteca dinâmica. A partir disso, produzir a mesma biblioteca para outras linguagens compiladas e interpretadas. Assim, para outros usuários, perde-se a dependência de uma linguagem e ganha-se desempenho, entretanto, aumentaria-se a dificuldade de manutenção da biblioteca.

Também é possível estudar usos da biblioteca para jogos ou situações que podem ser aplicadas a gamificação, como casos em que um indivíduo não possui função \fitness do tipo $f(x)$, mas que pode ser comparado com outros indivíduos.