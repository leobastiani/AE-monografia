O algoritmo desenvolvido pode ser aplicado para diversas áreas da computação bio-inspirada e pode-se acrescentar um novo conceito sobre funções \fitness. A função \fitness pode ser obtida em casos de disputa entre dois jogadores comparando o desempenho entre eles numa disputa. Isso cria outro paradigma na produção de um \SE, entre a comparação de melhores ou piores soluções.

Outra alternativa de uso da biblioteca é encontrar uma solução que não seja a ideal de um sistema, mas sim uma solução que vença outra solução fornecida, ou seja, um possível uso da biblioteca é não confrontar indivíduos entre si, mas sim confrontar um indivíduo com outro fornecido pelo usuário como uma solução base.

Como visto nas figuras \ref{figura:resultado_97} e \ref{figura:resultado_79}, a solução encontrado pela biblioteca é fruto de um processo de convergência do algoritmo evolutivo, mesmo podendo haver resultados em que uma solução totalmente aleatória fica com valores próximos a solução do \SE.

\section{Contribuições}

Este trabalho contribui para a comunidade por disponibilizar uma biblioteca para algoritmos bio-inspirados de código-aberto e gratuita, sua eficácia foi testada por este mesmo projeto, deixando evidente como a ferramenta pode impactar em uma solução requerida.

Para o autor, o projeto contribui em seu arsenal de apresentações, dando mais uma oportunidade de poder mostrar os conhecimentos adquiridos na graduação, tendo em mãos um projeto inusitado com resultados que muitos não imaginariam que seria capaz de se obter.

\section{Relacionamento entre o Curso e o Projeto}

O Curso contribuiu principalmente com a disciplina de Algoritmos Evolutivos ministrada pelo professor Eduardo Simões do Valle que foi capaz de inspirar e libertar a imaginação dos alunos em aula. Em segundo lugar, é evidente que a disciplina de Programação Orientadas a Objetos tem um papel importantíssimo, já que este projeto implementa os conceitos aprendidos nesta disciplina e se preocupa com reutilização do código disponibilizado.

Por fim, como dito anteriormente, a preocupação sobre a reutilização do código faz com que esse projeto também esteja relacionado com as disciplinas de Introdução a Ciências da Computação, pois as técnicas aplicadas na criação da biblioteca e a complexidade da mesma foram assuntos abordados desde o início, até a conclusão sobre a viabilidade da ferramenta.

\section{Considerações sobre o Curso de Graduação}

Para o autor, o curso possui disciplinas que não aproximam um engenheiro de computação da computação como obrigatórios e possui poucas disciplinas que preparam para o mercado de trabalho. Faltam disciplinas com planejamento de projetos que associam \textit{hardware} e \textit{software}. Na opinião do autor, o engenheiro da computação tem como disciplinas chaves Aplicação de Microprocessadores I e II.

\section{Trabalhos futuros}

Para remover a dependência da linguagem Java, a biblioteca produzida pode ser escrita em C/C++ e compilada como uma biblioteca dinâmica. A partir disso, produzir a mesma biblioteca para outras linguagens compiladas e interpretadas. Assim, para outros usuários, perde-se a dependência de uma linguagem e ganha-se desempenho, entretanto, aumentaria-se a dificuldade de manutenção da biblioteca.

Também é possível estudar usos da biblioteca para jogos ou situações que podem ser aplicadas a gamificação, como casos em que um indivíduo não possui função \fitness do tipo $f(x)$, mas que pode ser comparado com outros indivíduos.