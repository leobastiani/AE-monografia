\section{Motivação e Contextualização}

\newcommand{\SE}{Sistema Evolutivo (SE)\xspace}

Este trabalho relata o desenvolvimento de uma biblioteca que aplica os conceitos de Computação Bioinspirada e sua aplicação em um sistema com entradas numéricas e com três tipos saídas bem definidas, em que o algoritmo tenta prever as saídas baseando-se nas entradas. O uso de algoritmos evolutivos é uma alternativa a algoritmos determinísticos  e incorporam outro viés de solução por um \sigla{SE}{Sistema Evolutivo} \cite{Layzell1999} para ajustar continuamente os parâmetros de controlador às variações na configuração do ambiente de trabalho.

A área de pesquisa conhecida como Computação Bioinspirada procura encontrar soluções elegantes e eficientes que resolvem problemas grandes que poderiam demandar tempo computacional incompatível com a necessidade de resposta por meio de técnicas tradicionais. O comportamento social de formigas e abelhas, estratégia de caça de predadores ou o ciclo de atividade e hibernação de ursos são modelos e inspirações para essa área. Estes exemplos ilustram soluções de problemas que podem incluir otimização, orientação e reconhecimento de padrões, tudo isso sendo obtido através de um processo evolutivo.\cite{Simoes2000}

A técnica estudada neste trabalho denomina-se Computação Evolutiva da classe de técnicas conhecida como Computação Bioinspirada, Biológica ou Biocomputação que se inspira em princípios biológicos para projetar o algoritmo computacional, mais especificamente no comportamento social de organismos.

Um algoritmo evolutivo é capaz de caminhar a solução de um problema através da comparação de uma função \fitness que é uma função de comparação entre duas soluções. Por tanto, nesse trabalho, a ferramenta desenvolvida tem sua função \fitness como a comparação entre duas entidades que competem, o vencedor é aquele com uma melhor função \fitness. Foi estudada uma aplicação desse algoritmo com a comparação entre um sistema com um vetor de entradas numéricas e um vetor de saídas bem definidas de modo a prever as saídas do sistema. Portanto, a competição estudada gira em torno do algoritmo que consegue prever mais saídas baseando-se somente nas entradas.

O sistema estudado pode ser abstraído da seguinte forma:
\begin{itemize}
    \item A cada rodada são gerados 3 números aleatórios, se o último número for maior que os dois primeiros, o sistema responde com o número 2.
    \item Há duas colunas que são preenchidas respectivamente com o primeiro e segundo número aleatório da rodada, se a somatória da primeira coluna for maior ou igual que a somatória da segunda coluna, o sistema responde como 0, se não, responde como 1.
\end{itemize}

\section{Objetivos}

Este trabalho visa desenvolver uma biblioteca de algoritmo evolutivo que se enquadra em situações nas quais pode-se comparar duas soluções por meio de disputas capazes de se adaptar às alterações de cada cenário e estudar sua aplicação em um sistema com entradas e saídas bem definidas de modo a prever novas saídas.

A biblioteca implementa os conceitos de Indivíduo, que é um dos participantes de uma disputa, e de Gene, que é um dos fatores que determinará como um indivíduo responde a uma entrada. Nos estudos da utilização dessa ferramenta, foram feitos indivíduos de acordo com a biblioteca desenvolvida que segue um sistema evolutivo e os resultados foram comparados com um sistema que gera indivíduos aleatoriamente, ou seja, não seguem um sistema evolutivo, apenas tentam encontrar uma respostas gerando valores aleatórios. Isso faz uma boa comparação para determinar se os resultados são frutos de um processo de um sistema evolutivo ou se são frutos de um processo aleatório descoordenado.

\section{Organização}

No Capítulo 2 é apresentado conceitos de programação bio-inspirada e técnicas que prevaleceram para a construção deste projeto. A seguir, no Capítulo 3, descrevem-se as atividades realizadas para a construção da biblioteca de algoritmo evolutivo baseada em disputa e a extensão e aplicação desta biblioteca em um caso de estudo. Finalmente, no Capítulo 4, apresentam-se as conclusões sobre as aplicações da biblioteca e sobre os resultados do caso de estudo, além de tratar sobre trabalhos futuros.
