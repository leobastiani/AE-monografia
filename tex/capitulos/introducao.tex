\section{Motivação e Contextualização}

\newcommand\SE{\sigla{SE}{Sistema Evolutivo}\xspace}

Este trabalho relata o desenvolvimento de uma biblioteca que aplica os conceitos de Computação Bioinspirada e sua aplicação em um sistema desconhecido com entradas e saídas bem definidas. O uso de algoritmos evolutivos é uma alternativa a algoritmos determinísticos  e incorporam outro viés de solução por um \SE\cite{Layzell1999} para ajustar continuamente os parâmetros de controlador às variações na configuração do ambiente de trabalho.

A área de pesquisa conhecida como Computação Bioinspirada procura encontrar soluções elegantes e eficientes que resolvem problemas grandes que poderiam demandar tempo computacional incompatível com a necessidade de resposta por meio de técnicas tradicionais. O comportamento social de formigas e abelhas, estratégia de caça de predadores ou o ciclo de atividade e hibernação de ursos são modelos e inspirações para essa área, estes exemplos ilustram soluções de problemas que podem incluir otimização, orientação e reconhecimento de padrões, tudo isso sendo obtido através de um processo evolutivo.\cite{Simoes2000}

A técnica estudada neste trabalho denomina-se Computação Evolutiva da classe de técnicas conhecida como Computação Bioinspirada, Biológica ou Biocomputação que se inspira em princípios biológicos para projetar o algoritmo computacional, mais especificamente no comportamento social de organismos.

Um algoritmo evolutivo é capaz de caminhar a solução de um problema através da comparação de uma função \fitness que é uma função de comparação entre duas soluções. Por tanto, nesse trabalho, a ferramenta desenvolvida tem sua função \fitness como a comparação entre duas entidades que competem, o vencedor é aquele com uma melhor função \fitness. Foi estudada uma aplicação desse algoritmo com a comparação entre um sistema com um vetor de entradas numéricas e um vetor de saídas bem definidas de modo a prever as saídas do sistema.

\section{Objetivos}

Este trabalho visa desenvolver uma biblioteca de algoritmo evolutivo que se enquadra em situações nas quais pode-se comparar duas soluções por meio de disputas capazes de se adaptar às alterações de cada cenário e estudar sua aplicação em um sistema com entradas e saídas bem definidas de modo a prever novas saídas.

A biblioteca implementa os conceitos de Indivíduo, que é um dos participantes de uma disputa, e de Gene, que é um dos fatores que determinará como um indivíduo responde a uma entrada. Nos estudos da utilização dessa ferramenta, foram feitos sistemas com saídas que seguem uma determinada lógica e os resultados foram comparados com um sistema que gera saídas aleatoriamente.

\section{Organização}

Este documento relata os métodos desenvolvidos e aplicados para a construção do algoritmo, além de citar fontes que comprovam a eficácia e estudos com aplicações do algoritmo que o validam.

O código-fonte produzido para essa aplicação está disponibilizado\footnote{Disponível em \url{https://github.com/leobastiani/AE}} e uso da biblioteca para novas ferramentas que podem ser modeladas nos casos propostos, isto é, que são capazes de realizar disputas entre dois, gerando um melhor vencedor e um perdedor.
