% ------------------------------------------------------------------------
% ------------------------------------------------------------------------
% ICMC: Modelo de Trabalho Acadêmico (tese de doutorado, dissertação de
% mestrado e trabalhos monográficos em geral) em conformidade com 
% ABNT NBR 14724:2011: Informação e documentação - Trabalhos acadêmicos -
% Apresentação
% ------------------------------------------------------------------------
% ------------------------------------------------------------------------

% Opções: 
%   Qualificação         = qualificacao 
%   Curso                = doutorado/mestrado
%   Situação do trabalho = pre-defesa/pos-defesa (exceto para qualificação)
% -- opções do pacote babel --
% Idioma padrão = brazil
	%french,	    	% idioma adicional para hifenização
	%spanish,			% idioma adicional para hifenização
	%english,			% idioma adicional para hifenização
	%brazil				% o último idioma é o principal do documento
\documentclass[brazil]{packages/icmc}

% Tag <hr> como comando \hr
\newcommand\hr{\par\vspace{-.5\ht\strutbox}\noindent\hrulefill\par}

% Comando simples para exibir comandos Latex no texto
\newcommand{\comando}[1]{\textbf{$\backslash$#1}}
% métodos em java
\newcommand{\method}[1]{\textit{#1}}

% ---
% Pacotes Opcionais
% ---
\usepackage{rotating}           % Usado para rotacionar o texto
\usepackage[all,knot,arc,import,poly]{xy}   % Pacote para desenhos gráficos
% Este pacote pode conflitar com outros pacotes gráficos como o ``pictex''
% Então é necessário usar apenas um dos pacotes conflitantes


% ---
% Informações de dados para CAPA e FOLHA DE ROSTO
% ---
\titulo{Algoritmo evolutivo para treinar um sistema inteligente para jogos de disputa}
\autor[Bastiani, L. G.]{Leonardo Guarnieri de Bastiani}
\orientador[Orientador:]{Prof. Dr.}{Eduardo do Valle Simões}
%\coorientador{Prof. Dr.}{Fulano de tal}
\curso{Engenharia de Computação}
\area{Computação Bioinspirada} % Área de concentração do trabalho
\data{08}{11}{2018} % Data do depósito
% ---


% ---
% RESUMOS
% ---

% Resumo em português
% conter no máximo 500 palavras
\newcommand{\fitness}{\textit{fitness}\xspace}
\textoresumo{
    Este trabalho trata do desenvolvimento de um algoritmo bioinspirado, composto por uma biblioteca e um estudo a partir dela, capaz de treinar um sistema inteligente para aprender a emular comportamento em jogos de disputa e de uma possível utilização desse algoritmo. Nos jogos do contexto deste algoritmo, há decisões a serem tomadas por dois jogadores, dependendo da situação de cada jogada. Para isso, o algoritmo evolutivo ajusta os parâmetros que são usados na tomada de decisão. Para o algoritmo, a função \fitness é a comparação dos resultados apresentados pelas soluções que forem sendo geradas para selecionar as mais adequadas para gerar novas alternativas de controladores.
    }{Computação bioinspirada, algoritmo evolutivo, aprendizado de máquina, gamificação}

% ---
% resumo em inglês
% ---
\textoresumo[english]{
    This project is about the development of a bioinspired algorithm capable of training an intelligent system to learn how to emulate behavior in dispute games and a possible use of this algorithm. In the games in the context of this algorithm, there are decisions to be made by two players, depending on the situation of each move. The evolutionary algorithm adjusts the parameters that are used in the decision making. The \fitness function is the comparison of the results presented by the solutions that are generated to select the most suitable ones to generate new alternatives of controllers.
    }{Bio-inspired computation, evolutionary algorithm, machine learning, gamification}
    
% ---
% Configurações de aparência do PDF final
% ---
% alterando o aspecto da cor azul
\definecolor{blue}{RGB}{41,5,195}

% informações do PDF
\makeatletter
\hypersetup{
     	pagebackref=true,
		pdftitle={\@title}, 
		pdfauthor={\@author},
    	pdfsubject={\imprimirpreambulo},
	    pdfcreator={LaTeX with abnTeX2/ICMC-USP},
		pdfkeywords={\palavraschave}, 
		colorlinks=true,       		% false: boxed links; true: colored links
    	linkcolor=blue,          	% color of internal links
    	citecolor=blue,        		% color of links to bibliography
    	filecolor=magenta,      	% color of file links
		urlcolor=blue,
		bookmarksdepth=4
}
\makeatother
% --- 

% ----------------------------------------------------------
% ELEMENTOS PRÉ-TEXTUAIS
% ----------------------------------------------------------

% Inserir a ficha catalográfica
%\incluifichacatalografica[tex/fichaCatalografica.pdf]
\incluifichacatalografica


% Inserir folha de aprovação
%\input{tex/folha-aprovacao}

% DEDICATÓRIA / AGRADECIMENTO / EPÍGRAFE
\textodedicatoria*{tex/pre-textual/dedicatoria}
\textoagradecimentos*{tex/pre-textual/agradecimentos}
\textoepigrafe*{tex/pre-textual/epigrafe}

% Inclui a lista de figuras
\incluilistadefiguras

% Inclui a lista de tabelas
%\incluilistadetabelas

% Inclui a lista de quadros
%\incluilistadequadros

% Inclui a lista de algoritmos
\incluilistadealgoritmos

% Inclui a lista de códigos
%\incluilistadecodigos

% Inclui a lista de siglas e abreviaturas
\incluilistadesiglas

% Inclui a lista de símbolos
%\incluilistadesimbolos

% ----
% Início do documento
% ----
\begin{document}

% ----------------------------------------------------------
% ELEMENTOS TEXTUAIS
% ----------------------------------------------------------
\textual

\chapter{Introdução}
\label{chapter:introducao}
\section{Motivação e Contextualização}

\newcommand{\SE}{Sistema Evolutivo (SE)\xspace}

Este trabalho relata o desenvolvimento de uma biblioteca que aplica os conceitos de Computação Bioinspirada e sua aplicação em um sistema com entradas numéricas e com três tipos saídas bem definidas, em que o algoritmo tenta prever as saídas baseando-se nas entradas. O uso de algoritmos evolutivos é uma alternativa a algoritmos determinísticos  e incorporam outro viés de solução por um \sigla{SE}{Sistema Evolutivo} \cite{Layzell1999} para ajustar continuamente os parâmetros de controlador às variações na configuração do ambiente de trabalho.

A área de pesquisa conhecida como Computação Bioinspirada procura encontrar soluções elegantes e eficientes que resolvem problemas grandes que poderiam demandar tempo computacional incompatível com a necessidade de resposta por meio de técnicas tradicionais. O comportamento social de formigas e abelhas, estratégia de caça de predadores ou o ciclo de atividade e hibernação de ursos são modelos e inspirações para essa área. Estes exemplos ilustram soluções de problemas que podem incluir otimização, orientação e reconhecimento de padrões, tudo isso sendo obtido através de um processo evolutivo.\cite{Simoes2000}

A técnica estudada neste trabalho denomina-se Computação Evolutiva da classe de técnicas conhecida como Computação Bioinspirada, Biológica ou Biocomputação que se inspira em princípios biológicos para projetar o algoritmo computacional, mais especificamente no comportamento social de organismos.

Um algoritmo evolutivo é capaz de caminhar a solução de um problema através da comparação de uma função \fitness que é uma função de comparação entre duas soluções. Por tanto, nesse trabalho, a ferramenta desenvolvida tem sua função \fitness como a comparação entre duas entidades que competem, o vencedor é aquele com uma melhor função \fitness. Foi estudada uma aplicação desse algoritmo com a comparação entre um sistema com um vetor de entradas numéricas e um vetor de saídas bem definidas de modo a prever as saídas do sistema. Portanto, a competição estudada gira em torno do algoritmo que consegue prever mais saídas baseando-se somente nas entradas.

O sistema estudado pode ser abstraído da seguinte forma:
\begin{itemize}
    \item A cada rodada são gerados 3 números aleatórios, se o último número for maior que os dois primeiros, o sistema responde com o número 2.
    \item Há duas colunas que são preenchidas respectivamente com o primeiro e segundo número aleatório da rodada, se a somatória da primeira coluna for maior ou igual que a somatória da segunda coluna, o sistema responde como 0, se não, responde como 1.
\end{itemize}

\section{Objetivos}

Este trabalho visa desenvolver uma biblioteca de algoritmo evolutivo que se enquadra em situações nas quais pode-se comparar duas soluções por meio de disputas capazes de se adaptar às alterações de cada cenário e estudar sua aplicação em um sistema com entradas e saídas bem definidas de modo a prever novas saídas.

A biblioteca implementa os conceitos de Indivíduo, que é um dos participantes de uma disputa, e de Gene, que é um dos fatores que determinará como um indivíduo responde a uma entrada. Nos estudos da utilização dessa ferramenta, foram feitos indivíduos de acordo com a biblioteca desenvolvida que segue um sistema evolutivo e os resultados foram comparados com um sistema que gera indivíduos aleatoriamente, ou seja, não seguem um sistema evolutivo, apenas tentam encontrar uma respostas gerando valores aleatórios. Isso faz uma boa comparação para determinar se os resultados são frutos de um processo de um sistema evolutivo ou se são frutos de um processo aleatório descoordenado.

\section{Organização}

No Capítulo 2 é apresentado conceitos de programação bio-inspirada e técnicas que prevaleceram para a construção deste projeto. A seguir, no Capítulo 3, descrevem-se as atividades realizadas para a construção da biblioteca de algoritmo evolutivo baseada em disputa e a extensão e aplicação desta biblioteca em um caso de estudo. Finalmente, no Capítulo 4, apresentam-se as conclusões sobre as aplicações da biblioteca e sobre os resultados do caso de estudo, além de tratar sobre trabalhos futuros.


\chapter{Métodos, Técnicas e Tecnologias Utilizadas}
\label{chapter:metodos}
\section{Revisão Bibliográfica}

A linguagem de programação Java foi escolhida para a criação da ferramenta, os motivos para essa escolha foram:

\newcommand{\JVM}{\sigla{JVM}{Java Virtual Machine}}

\begin{itemize}
    \item Multiplataforma: Java é uma linguagem capaz de ser executada em qualquer sistema que possua uma \JVM, sua execução e desenvolvimento independe do sistema, permitindo que a aplicação desenvolvida não sofra essa restrição.
    \item Desempenho: entre as linguagens de programação que são interpretadas, Java possui um desempenho melhor por ser compilada e interpretada, gerando um código intermediário que é executado pela \JVM.
    \item Facilidade e didática: Java é uma linguagem que permite certas comodidades ao programador com um bom balanço de desempenho, como a ferramenta desenvolvida foi utilizada para estudos e é de uso genérico, as facilidades fornecidas pela linguagem são importante para agilizar e pela validação do trabalho.
    \item Modelagem: a modelagem de classes fornecida por Java ajudam no entendimento e no reuso da biblioteca desenvolvida.
\end{itemize}

Nesse projeto há uma contextualização do problema nos seguintes moldes:

\begin{itemize}
    \item Indivíduo: um indivíduo é a representação de um jogador que possui uma pontuação provida da função \fitness de acordo com as respostas produzidas por esse jogador.
    \item Gene: um gene é uma entidade capaz de provocar uma resposta. Para cada tipo de respostas há um grupo de genes.
    \item Disputa: é uma função entre dois indivíduos que gera um vencedor e um perdedor com base na função \fitness. Uma disputa deve necessariamente retornar o quão melhor um indivíduo foi em relação a outro.
\end{itemize}

Por tanto, o indivíduo que possui os melhores genes terá uma função \fitness melhor e ganhará mais disputas.

A inspiração do paradigma da Computação Evolutiva vem da evolução natural formalizada por Darwin\footnote
{Ridley , M. (1996). Evolution. Blackwell Science.}.
Algoritmos que embebidos desta teste possuem passos genéricos capazes de resolver um grande número de problemas práticos, e possuem caractéristicas como auto-organização e comportamento adaptativo\footnote
{Goldberg, D. E. and Holland, J. H. (1988). Genetic Algorithms and Machine Learning: Introduction to the Special Issue on Genetic Algorithms. Machine Learning.},
portanto são capazes de tratar problemas computacionalmente complexos com uma ferramenta de propósito geral. Por outro lado, esses algoritmos não garantem a obtenção de uma solução ótima\footnote
{Zuben, F . J. V . (2000). Computação evolutiva: Uma abordagem pragmática. Anais da 1ª Jornada de Estudos em Computação de Piracicaba e região (1ª JECOMP), 1:25-45.}
por poder convergir para soluções localmente ótimas.

\section{Algoritmos evolutivos}
Os algoritmos evolutivos podem ser resumidos em um roteiro básico de procedimentos\footnote
{Todd, P. M. e Miller, G. F., (1997) Biodiversity Through Sexual Selection. In Artificial Life V, Langton, C. G. and Shimohara, K. (Eds.), Publisher: MIT Press, Cambridge, MA. USA, pp. 289-299.}
que são adaptados de acordo com o contexto do problema a ser solucionado\footnote
{Werger, B. B. e Mataric, M. J. (1999) Exploiting Embodiment in Multi-Robot Teams. Report ID: Technical Report IRIS-99-378, University of Southern California, Institute for Robotics and Intelligent Systems, 14p}\footnote
{Mitchell, M. (1995) Genetic Algorithms and Artificial Life. In Artificial Life An Overview, Langton, C. G. (Ed.), Publisher: MIT Press, pp. 267-289.}.
Um algoritmo evolutivo é usualmente composto por:

\begin{itemize}
    \item Representação de genes dos indivíduos: Biologicamente, cada gene representa uma característica do indivíduo, e o cromossomo, que é o conjunto de genes de um indivíduo, representa uma o indivíduo. Computacionalmente, o cromossomo representa um candidato à solução do problema e o gene representa uma função para obtenção de uma resposta.
    \item Função \fitness (ou de adaptação): indica, para cada indivíduo, o valor de aptidão, mostrando a aproximação da solução proposta pelo indivíduo em relação à solução procurada.
    \item Função de reprodução: Determina novos indivíduos que representarão a próxima geração da população do algoritmo herdando características positivas que são encontradas nos indivíduos com uma maior função \fitness (ou mais aptos), propagando seus genes. O crossover e mutações ocorrem na função de reprodução, de modo análogo ao biológico.
\end{itemize}

\chapter{Desenvolvimento}
\label{chapter:desenvolvimento}
\section{O Problema}

O desenvolvimento de um algoritmo evolutivo genérico para solucionar problemas com gamificação de disputas, ou seja, soluções que podem ser qualificadas entre o quão melhores uma é das demais.

Estudar as aplicações desse algoritmo em um sistema com entradas e saídas bem definidas, de modo que a solução não esteja implícita em nenhuma das entradas e saídas, mas que o algoritmo possa evoluir livremente para encontrar o melhor rearranjo dos genes até produzir respostas idênticas as oferecidas pelo sistema.

\section{Atividades Realizadas}

Foi feita uma biblioteca para uso de algoritmos evolutivos em Java que empregas os conceitos de Indivíduos e Genes. Para aplicações diferentes das estudadas nesse documento, a biblioteca ainda tem um uso muito bom, pois pode ser estendida e rearranjada para qualquer aplicação dentro desse conceito.

\section{Resultados}

A biblioteca produzida pode ser aplicada para quaisquer situações em que a gamificação baseada em disputas de dois jogadores pode ser aplicada.

Em sistemas cujas respostas não são determinísticas, mas sabe-se as entradas e as respostas, este algoritmo pode ser aplicado utilizando o conceito de Genes desenvolvido. Os resultados podem não ser exatos, porém haverá respostas mais consistentes do que uma resposta aleatória.

\section{Dificuldades e Limitações}

A maior dificuldade desse algoritmo, assim como da maioria dos algoritmos evolutivos, é encontrar uma função \fitness capaz de representar o quão uma solução é melhor do que outra.

\chapter{Conclusão}
\label{chapter:conclusao}
O algoritmo desenvolvido pode ser aplicado para diversas áreas da computação bio-inspirada e acrescenta um novo conceito sobre funções \fitness. A função \fitness pode ser obtida em casos de disputa entre dois jogadores comparando o desempenho entre eles numa disputa. Isso cria outro paradigma na criação de um \SE, entre a comparação de melhores ou piores soluções.

Há outra alternativa, o algoritmo pode não encontrar a solução ideal de um sistema, mas pode encontrar uma solução que vença uma solução fornecida, ou seja, um possível uso da biblioteca é não confrontar indivíduos entre si, mas sim confrontar um indivíduo com outro fornecido pelo usuário como uma solução base.

Como visto nas figuras \ref{figura:resultado_97} e \ref{figura:resultado_79}, a solução encontrado pela biblioteca é fruto de um processo de convergência do algoritmo evolutivo, mesmo podendo haver resultados em que uma solução totalmente aleatória fica com valores próximos a solução do \SE.

% \chapter{Citações e Referências}
% \label{chapter:citacoes}
% \input{tex/citacoes}


% ---
% Finaliza a parte no bookmark do PDF, para que se inicie o bookmark na raiz
% ---
\bookmarksetup{startatroot}% 
% ---

% ----------------------------------------------------------
% ELEMENTOS PÓS-TEXTUAIS
% ----------------------------------------------------------
\postextual

% ----------------------------------------------------------
% Referências bibliográficas
% ----------------------------------------------------------
\bibliography{references}

% ---------------------------------------------------------------------
% GLOSSÁRIO
% ---------------------------------------------------------------------

% Arquivo que contém as definições que vão aparecer no glossário
\newword{30}{Função fitness}{Função que pode medir o quão boa é uma solução em uma escala numérica}

\newword{35}{Java}{Linguagem de programação orientada a objetos que gera um \textit{bytecode} para ser interpretado pela JVM, portanto, é uma linguagem semi-compilada, trazendo benefícios de linguagens compiladas e interpretadas}

\newword{36}{Java Virtual Machine}{JVM é uma máquina virtual que possibilita um computador executar programas em Java}

\newword{37}{AE}{Algoritmo Evolutivo. Algoritmos relacionados à Computação Evolutiva. Englobam os algoritmos Genéticos, mas não se resumem a estes. Inspirados na Teoria da Evolução Natural de Charles Darwin}

\newword{38}{Cromossomo}{Solução candidata, utilizada pelo Algoritmo Evolutivo em seu processo de busca}

\newword{39}{Crossover}{Mecanismo da biologia abstraído na forma de uma função matemática utilizada pelo Algoritmo Evolutivo, aplicada sobre os cromossomos. Faz com que uma nova solução possua características de soluções anteriores}

\newword{40}{Adaptação}{Melhoria do valor de aptidão (fitness) presente no passar das gerações}

\newword{41}{Aptidão}{Pontuação que indica a eficácia da solução encontrada}

\newword{42}{Geração}{O conjunto das soluções contidas em uma população em um determinado instante; sempre que os operadores evolutivos criam uma nova população, com novas soluções, diz-se que uma nova geração foi iniciada}

\newword{43}{Mutação}{Mecanismo da biologia abstraído na forma de uma função matemática utilizada pelo Algoritmo Evolutivo, aplicada sobre os cromossomos. Insere variedade nas soluções da população}

% Comando para incluir todas as definições do arquivo glossario.tex
\glsaddall
% Impressão do glossário
\printglossaries

% ----------------------------------------------------------
% Apêndices
% ----------------------------------------------------------

% ---
% Inicia os apêndices
% ---
% \begin{apendicesenv}

%     \chapter{Documento Básico Usando a Classe icmc}
%     \label{chapter:documento-basico}
%     \input{tex/appendix/documento-basico}

% \end{apendicesenv}
% ---


% ----------------------------------------------------------
% Anexos
% ----------------------------------------------------------

% ---
% Inicia os anexos
% ---
% \begin{anexosenv}

%     \chapter{Páginas Interessantes na Internet} 
%     \label{chapter:paginas-interessantes}
%     \input{tex/annex/paginas-interessantes}

% \end{anexosenv}
% ---

\end{document}